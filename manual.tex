\documentclass[pdftex,10pt]{article}
\usepackage{amsmath,amsfonts,url}
\usepackage[pdftex]{graphicx}
\usepackage{booktabs}
\usepackage{url}
\setlength{\parindent}{0pt}
\def\cour{\fontfamily{pcr}\selectfont}
\usepackage[top=1in, bottom=0.8in, left=0.8in, right=0.8in]{geometry}
\setcounter{tocdepth}{3}
\newlength{\titrwidth}
\setlength{\titrwidth}{\textwidth}
\addtolength{\titrwidth}{-1.2in}
\newlength{\methwidth}
\setlength{\methwidth}{0.8in}
\newlength{\defnwidth}
\setlength{\defnwidth}{\textwidth}
\addtolength{\defnwidth}{-1.2in}
\addtolength{\heavyrulewidth}{\heavyrulewidth}
\def\qquad{\quad\quad}
\def\qqqquad{\quad\quad\quad\quad}
\def\cc{\colon\colon}
\def\gmpz{\textit{GMP::Z}}
\def\gmpq{\textit{GMP::Q}}
\def\gmpf{\textit{GMP::F}}
\def\gmprandstate{\textit{GMP::RandState}}
\def\gmpzs{\textit{GMP::Z}\ }
\def\gmpqs{\textit{GMP::Q}\ }
\def\gmpfs{\textit{GMP::F}\ }
\def\gmprandstates{\textit{GMP::RandState\ }}
\frenchspacing
\begin{document}

\begin{tabular}{p{1.0in} p{\titrwidth}}
\huge{gmp} &\\
\midrule[3pt]
\multicolumn{2}{r}{\large{Ruby bindings to the GMP library}}\\
\multicolumn{2}{r}{\large{Edition 0.7.19}}\\
\multicolumn{2}{r}{\large{12 March 2014}}
\end{tabular}

\vfill
\large{written by Sam Rawlins}\\
\large{with extensive quoting from the GMP Manual}
\newpage

\vfill
This manual describes how to use the gmp Ruby gem, which provides bindings to
the GNU multiple precision arithmetic library, version 4.3.x or 5.x.\\
\\
Copyright 2009, 2010, 2011, 2012, 2013 Sam Rawlins.\\
Apache License, Version 2.0
\newpage

\tableofcontents
\newpage

\section{Introduction to GNU MP}

This entire page is copied verbatim from the GMP Manual.\\\\

GNU MP is a portable library written in C for arbitrary precision arithmetic on
integers, rational numbers, and floating-point numbers. It aims to provide the
fastest possible arithmetic for all applications that need higher precision
than is directly supported by the basic C types.\\
\\
Many applications use just a few hundred bits of precision; but some
applications may need thousands or even millions of bits. GMP is designed to
give good performance for both, by choosing algorithms based on the sizes of
the operands, and by carefully keeping the overhead at a minimum.\\
\\
The speed of GMP is achieved by using fullwords as the basic arithmetic type,
by using sophisticated algorithms, by including carefully optimized assembly
code for the most common inner loops for many different CPUs, and by a general
emphasis on speed (as opposed to simplicity or elegance).\\
\\
There is assembly code for these CPUs: ARM, DEC Alpha 21064, 21164, and 21264,
AMD 29000, AMD K6, K6-2, Athlon, and Athlon64, Hitachi SuperH and SH-2, HPPA
1.0, 1.1, and 2.0, Intel Pentium, Pentium Pro/II/III, Pentium 4, generic x86,
Intel IA-64, i960, Motorola MC68000, MC68020, MC88100, and MC88110,
Motorola/IBM PowerPC 32 and 64, National NS32000, IBM POWER, MIPS R3000, R4000,
SPARCv7, SuperSPARC, generic SPARCv8, UltraSPARC, DEC VAX, and Zilog Z8000.
Some optimizations also for Cray vector systems, Clipper, IBM ROMP (RT), and
Pyramid AP/XP.\\
\\
For up-to-date information on GMP, please see the GMP web pages at 
\url{http://gmplib.org/}\\

The latest version of the library is available at 
\url{ftp://ftp.gnu.org/gnu/gmp/}\\

Many sites around the world mirror '\url{ftp.gnu.org}', please use a mirror
near you, see \url{http://www.gnu.org/order/ftp.html} for a full list.\\
\\
There are three public mailing lists of interest. One for release
announcements, one for general questions and discussions about usage of the GMP
library, and one for bug reports. For more information, see
\url{http://gmplib.org/mailman/listinfo/}.\\

The proper place for bug reports is gmp-bugs@gmplib.org. See Chapter 4
[Reporting Bugs], page 28 for information about reporting bugs.

\newpage
\section{Introduction to MPFR}

The gmp gem optionally interacts with the MPFR library as well. This entire page
is copied verbatim from the MPFR manual.\\\\

The MPFR library is a C library for multiple-precision floating-point
computations with correct rounding. MPFR has continuously been supported by the
INRIA and the current main authors come from the Caramel and Ar�naire
project-teams at Loria (Nancy, France) and LIP (Lyon, France) respectively; see
more on the credit page. MPFR is based on the GMP multiple-precision library.\\

The main goal of MPFR is to provide a library for multiple-precision
floating-point computation which is both efficient and has a well-defined
semantics. It copies the good ideas from the ANSI/IEEE-754 standard for
double-precision floating-point arithmetic (53-bit mantissa).\\

MPFR is free. It is distributed under the GNU Lesser General Public License
(GNU Lesser GPL), version 3 or later (2.1 or later for MPFR versions until
2.4.x). The library has been registered in France by the Agence de Protection
des Programmes under the number IDDN FR 001 120020 00 R P 2000 000 10800, on 15
March 2000. This license guarantees your freedom to share and change MPFR, to
make sure MPFR is free for all its users. Unlike the ordinary General Public
License, the Lesser GPL enables developers of non-free programs to use MPFR in
their programs. If you have written a new function for MPFR or improved an
existing one, please share your work!

\newpage
\section{Introduction to the gmp gem}

The gmp Ruby gem is a Ruby library that provides bindings to GMP. The gem is
incomplete, and will likely only include a subset of the GMP functions. It is
built as a C extension for Ruby, interacting with gmp.h. The gmp gem is not
endorsed or supported by GNU or the GMP team (or MPFR team). The gmp gem also
does not ship with GMP (or MPFR), so GMP (and MPFR) must be compiled
separately.

\section{Installing the gmp gem}

\subsection{Prerequisites}
OK. First, we've got a few requirements. To install the gmp gem, you need one
of the following versions of Ruby:
\begin{itemize}
  \item (MRI) Ruby 1.8.6 - tested lightly.
  \item (MRI) Ruby 1.8.7 - tested lightly.
  \item (MRI) Ruby 1.9.3 - tested seriously.
  \item (MRI) Ruby 2.0.0 - tested seriously.
  \item (REE) Ruby 1.8.7 - tested lightly.
  \item (RBX) Rubinius 1.1 - tested lightly.
\end{itemize}
As you can see only Matz's Ruby Interpreter (MRI) is seriously supported. I've just
started to poke around with REE. Everything seems to work on REE 1.8.7 on Linux, x86 and
x86\_64. Also, Rubinius 1.1 seems to work great on Linux, but support won't be official
until Rubinius 1.1.1.\\

Next is the platform, the combination of the architecture (processor) and OS.
As far as I can tell, if you can compile GMP and Ruby (and optionally MPFR) on
a given platform, you can use the gmp gem there too. Please report problems
with that hypothesis.\\

Lastly, GMP (and MPFR). GMP (and MPFR) must be compiled and working. "And working" means
you ran "make check" after compiling GMP (and MPFR), and it 'check's out. The following
versions of GMP (and MPFR) have been tested:
\begin{itemize}
  \item GMP 4.3.1 (with MPFR 2.4.2)
  \item GMP 4.3.2 (with MPFR 2.4.2 and 3.0.0)
  \item GMP 5.0.0 (with MPFR 3.0.0)
  \item GMP 5.0.1 (with MPFR 3.0.0)
\end{itemize}

That's all. I don't intend to test any older versions.\\

\newpage
Here is a table of the exact environments on which I have tested the gmp gem.
The (MPFR) version denotes that the gmp gem was tested both with and without
the given version of MPFR:\\\\

\begin{tabular}{lrrr} \hline
             Platform                      &       Ruby         &    GMP    & (MPFR)  \\ \midrule[1pt]
  Linux (Ubuntu NR 10.04) on x86 (32-bit)  & (MRI) Ruby 1.8.7   & GMP 4.3.2 & (2.4.2) \\
                                           & (MRI) Ruby 1.8.7   & GMP 5.0.1 & (3.0.0) \\
                                           & (MRI) Ruby 1.9.1   & GMP 4.3.2 & (2.4.2) \\
                                           & (MRI) Ruby 1.9.1   & GMP 5.0.1 & (3.0.0) \\
                                           & (MRI) Ruby 1.9.2   & GMP 4.3.2 & (2.4.2) \\
                                           & (MRI) Ruby 1.9.2   & GMP 5.0.1 & (3.0.0) \\
                                           & (RBX) Rubinius 1.1 & GMP 4.3.2 & (2.4.2) \\
                                           & (RBX) Rubinius 1.1 & GMP 5.0.1 & (3.0.0) \\ \hline
  Linux (Ubuntu 10.04) on x86\_64 (64-bit) & (MRI) Ruby 1.8.7   & GMP 4.3.2 & (2.4.2) \\
                                           & (MRI) Ruby 1.8.7   & GMP 5.0.1 & (3.0.0) \\
                                           & (MRI) Ruby 1.9.1   & GMP 4.3.2 & (2.4.2) \\
                                           & (MRI) Ruby 1.9.1   & GMP 5.0.1 & (3.0.0) \\
                                           & (MRI) Ruby 1.9.2   & GMP 4.3.2 & (2.4.2) \\
                                           & (MRI) Ruby 1.9.2   & GMP 5.0.1 & (3.0.0) \\
                                           & (RBX) Rubinius 1.1 & GMP 4.3.2 & (2.4.2) \\
                                           & (RBX) Rubinius 1.1 & GMP 5.0.1 & (3.0.0) \\ \hline
  Mac OS X 10.6.4 on x86\_64 (64-bit)      & (MRI) Ruby 1.8.7   & GMP 4.3.2 & (2.4.2) \\
                                           & (MRI) Ruby 1.8.7   & GMP 5.0.1 & (3.0.0) \\
                                           & (MRI) Ruby 1.9.1   & GMP 4.3.2 & (2.4.2) \\
                                           & (MRI) Ruby 1.9.1   & GMP 5.0.1 & (3.0.0) \\
                                           & (MRI) Ruby 1.9.2   & GMP 4.3.2 & (2.4.2) \\
                                           & (MRI) Ruby 1.9.2   & GMP 5.0.1 & (3.0.0) \\
                                           & (RBX) Rubinius 1.1 & GMP 4.3.2 & (2.4.2) \\
                                           & (RBX) Rubinius 1.1 & GMP 5.0.1 & (3.0.0) \\ \hline
  Windows 7 on x86\_64 (64-bit)            & (MRI) Ruby 1.8.7   & GMP 4.3.2 & (2.4.2) \\
                                           & (MRI) Ruby 1.8.7   & GMP 5.0.1 & (3.0.0) \\
                                           & (MRI) Ruby 1.9.1   & GMP 4.3.2 & (2.4.2) \\
                                           & (MRI) Ruby 1.9.1   & GMP 5.0.1 & (3.0.0) \\
                                           & (MRI) Ruby 1.9.2   & GMP 4.3.2 & (2.4.2) \\
                                           & (MRI) Ruby 1.9.2   & GMP 5.0.1 & (3.0.0) \\ \hline
  Windows XP on x86 (32-bit)               & (MRI) Ruby 1.9.1   & GMP 4.3.2 & (2.4.2) \\
                                           & (MRI) Ruby 1.9.1   & GMP 5.0.1 & (3.0.0) \\ \hline
\end{tabular}\\\\

\newpage
In addition, I \textit{used to test} on the following environments, in versions
0.4.7 and earlier of the gmp gem:\\\\

\begin{tabular}{lrr} \hline
             Platform             &       Ruby       &    GMP    \\ \midrule[1pt]
  Cygwin on x86                   & (MRI) Ruby 1.8.7 & GMP 4.3.1 \\ \hline
  Linux (LinuxMint 7) on x86      & (MRI) Ruby 1.8.7 & GMP 4.3.1 \\ \hline
  Mac OS X 10.5.7 on x86 (32-bit) & (MRI) Ruby 1.8.6 & GMP 4.3.1 \\ \hline
  Mac OS X 10.5.7 on x86 (32-bit) & (MRI) Ruby 1.9.1 & GMP 4.3.1 \\ \hline
\end{tabular}

\subsection{Installing}
You may clone the gmp gem's git repository with:\\
\\
\texttt{git clone git://github.com/srawlins/gmp.git}\\

Or you may install the gem from gemcutter (rubygems.org):\\
\\
\texttt{gem install gmp}\\

\newpage
At this time, the gem self-compiles. If required libraries cannot be found, you may
compile the C extensions manually with:\\
\\
\texttt{cd <gmp gem directory>/ext}\\
\texttt{ruby extconf.rb}\\
\texttt{make}\\

There shouldn't be any errors, or warnings.

\section{Testing the gmp gem}

Testing the gmp gem is quite simple. The test/unit\_tests.rb suite uses Unit::Test.
You can run this test suite with:\\
\\
\texttt{cd <gmp gem directory>/test}\\
\texttt{ruby unit\_tests.rb}\\

All tests should pass. If you don't have the test-unit gem installed, then you may
run into one error. It would look like:\\
\\
\texttt{  1) Error:}\\
\texttt{test\_z\_div(TC\_division):}\\
\texttt{TypeError: GMP::Q can't be coerced into Float}\\
\texttt{    C:/Ruby191/devkit/msys/1.0.11/projects/gmp\_gem/test/tc\_division.rb:18:in `test\_z\_div'}\\

\section{GMP and gmp gem basics}

\subsection{Classes}
The gmp gem includes the namespace \texttt{GMP} and four classes within \texttt{GMP}:
\begin{itemize}
  \item \texttt{GMP::Z} - Methods for signed integer arithmetic. There are about 64
    methods here.
  \item \texttt{GMP::Q} - Methods for rational number arithmetic. There are at least 11
    methods here (still accounting).
  \item \texttt{GMP::F} - Methods for floating-point arithmetic. There are at least 6
    methods here (still accounting).
  \item \texttt{GMP::RandState} - Methods for random number generation. There are 3
    methods here.
\end{itemize}

In addition to the above four classes, there are also four constants within \texttt{GMP}:
\begin{itemize}
  \item \texttt{GMP::GMP\_VERSION} - The version of GMP linked into the gmp gem
  \item \texttt{GMP::GMP\_CC} - The compiler that compiled GMP linked into the gmp gem
  \item \texttt{GMP::GMP\_CFLAGS} - The compiler flags used to compile GMP linked into
    the gmp gem
  \item \texttt{GMP::GMP\_BITS\_PER\_LIMB} - The number of bits per limb
  \item \texttt{GMP::GMP\_NUMB\_MAX} - The maximum value that can be stored in the number
    part of a limb.
\end{itemize}

\section{MPFR basics}

The gmp gem can optionally link to MPFR, the Multiple Precision Floating-Point Reliable
Library. The x86-mswin32 version of the gmp gem comes with MPFR. This library uses the
floating-point type from GMP, and thus the MPFR functions mapped in the gmp gem become
methods in \texttt{GMP::F}.

There are additional constants within \texttt{GMP} when MPFR is linked:
\begin{itemize}
  \item \texttt{GMP::MPFR\_VERSION} - The version of MPFR linked into the gmp gem.
  \item \texttt{GMP::MPFR\_PREC\_MIN} - The minimum precision available.
  \item \texttt{GMP::MPFR\_PREC\_MAX} - The maximum precision available
  \item \texttt{GMP::GMP\_RNDN} - Rounding mode representing "round to nearest."
  \item \texttt{GMP::GMP\_RNDZ} - Rounding mode representing "round toward zero."
  \item \texttt{GMP::GMP\_RNDU} - Rounding mode representing "round toward positive
    infinity."
  \item \texttt{GMP::GMP\_RNDD} - Rounding mode representing "round toward negative
    infinity."
  \item \texttt{GMP::MPFR\_RNDN} - Rounding mode representing "round to nearest."\\
    (MPFR version 3.0.0 or higher only)
  \item \texttt{GMP::MPFR\_RNDZ} - Rounding mode representing "round toward zero."\\
    (MPFR version 3.0.0 or higher only)
  \item \texttt{GMP::MPFR\_RNDU} - Rounding mode representing "round toward positive
    infinity."\\
    (MPFR version 3.0.0 or higher only)
  \item \texttt{GMP::MPFR\_RNDD} - Rounding mode representing "round toward negative
    infinity."\\
    (MPFR version 3.0.0 or higher only)
  \item \texttt{GMP::MPFR\_RNDZ} - Rounding mode representing "round away from zero."\\
    (MPFR version 3.0.0 or higher only)
\end{itemize}

\newpage
\section{Integer Functions}

\subsection{Initializing, Assigning Integers}

\begin{tabular}{p{\methwidth} l r}
\toprule
\textbf{new} & & GMP::Z.new $\rightarrow$ \textit{integer} \\
& & GMP::Z.new(\textit{numeric = 0}) $\rightarrow$ \textit{integer} \\
& & GMP::Z.new(\textit{str}, \textit{base = 0}) $\rightarrow$ \textit{integer} \\
\cmidrule(r){2-3}
& \multicolumn{2}{p{\defnwidth}}{
  This method creates a new \gmpzs integer. It typically takes one optional argument for
  the value of the integer. This argument can be one of several classes. If the first
  argument is a String, then a second argument, the base, may be optionally supplied.
  Here are some examples:\newline

  \texttt{GMP::Z.new \qqqquad\qqqquad \#=> 0 (default) \newline
          GMP::Z.new(1) \qqqquad\qquad\  \#=> 1 (Ruby Fixnum) \newline
          GMP::Z.new("127") \qqqquad\  \#=> 127 (Ruby String)\newline
          GMP::Z.new("FF", 16) \qquad\ \  \#=> 255 (Ruby String with base)\newline
          GMP::Z.new("1Z", 36) \qquad\ \  \#=> 71 (Ruby String with base)\newline
          GMP::Z.new(4294967296) \qquad \#=> 4294967296 (Ruby Bignum)\newline
          GMP::Z.new(GMP::Z.new(31))  \#=> 31 (GMP Integer)}
}
\end{tabular}
\newline\newline

There is also a convenience method available, \texttt{GMP::Z()}.\\

\subsection{Converting Integers}

\begin{tabular}{p{\methwidth} l r}
\toprule
\textbf{to\_d} & & \textit{integer}.to\_d $\rightarrow$ \textit{float} \\
\cmidrule(r){2-3}
& \multicolumn{2}{p{\defnwidth}}{
  Returns \textit{integer} as an Float if \textit{integer} fits in a Float.

  Otherwise returns the least significant part of \texttt{integer}, with the same sign as
  \textit{integer}.
  
  If \textit{integer} is too big to fit in a Float, the returned result is probably not
  very useful. To find out if the value will fit, use the function
  \textit{mpz\_fits\_slong\_p} (\textbf{Unimplemented}).
}
\end{tabular}
\newline\newline

\begin{tabular}{p{\methwidth} l r}
\toprule
\textbf{to\_i} & & \textit{integer}.to\_i $\rightarrow$ \textit{fixnum} \\
\cmidrule(r){2-3}
& \multicolumn{2}{p{\defnwidth}}{
  Returns \textit{integer} as a Fixnum if \textit{integer} fits in a Fixnum.
  \newline
  
  Otherwise returns the least significant part of \textit{integer}, with the same sign as
  \textit{integer}.
  \newline
  
  If \textit{integer} is too big to fit in a \textit{Fixnum}, the returned result is
  probably not very useful. To find out if the value will fit, use the function
  \textit{mpz\_fits\_slong\_p} (\textbf{Unimplemented}).
}
\end{tabular}
\newline\newline

\begin{tabular}{p{\methwidth} l r}
\toprule
\textbf{to\_s} & & \textit{integer}.to\_s(\textit{base = 10}) $\rightarrow$ \textit{str} \\
\cmidrule(r){2-3}
& \multicolumn{2}{p{\defnwidth}}{
  Converts \textit{integer} to a string of digits in base \textit{base}. The
  \textit{base} argument may vary from 2 to 62 or from -2 to -36, or be a symbol, one of
  \textit{:bin}, \textit{:oct}, \textit{:dec}, or \textit{:hex}.
  \newline
  
  For \textit{base} in the range 2..36, digits and lower-case letters are used; for
  -2..-36 (and \textit{:bin}, \textit{:oct}, \textit{:dec}, and \textit{:hex}), digits
  and upper-case letters are used; for 37..62, digits, upper-case letters, and lower-case
  letters (in that significance order) are used. Here are some
  examples:\newline
  
  \texttt{GMP::Z(1).to\_s \qqqquad \#=> "1" \newline
          GMP::Z(32).to\_s(2) \qquad \#=> "100000" \newline
          GMP::Z(32).to\_s(4) \qquad \#=> "200" \newline
          GMP::Z(10).to\_s(16) \quad\  \#=> "a" \newline
          GMP::Z(10).to\_s(-16) \quad \#=> "A" \newline
          GMP::Z(255).to\_s(:bin) \#=> "11111111" \newline
          GMP::Z(255).to\_s(:oct) \#=> "377" \newline
          GMP::Z(255).to\_s(:dec) \#=> "255" \newline
          GMP::Z(255).to\_s(:hex) \#=> "ff"}
}
\end{tabular}

\subsection{Integer Arithmetic}

\begin{tabular}{p{\methwidth} l r}
\toprule
\textbf{+} & & \textit{integer} + \textit{numeric} $\rightarrow$ \textit{numeric} \\
\cmidrule(r){2-3}
& \multicolumn{2}{p{\defnwidth}}{
  Returns the sum of \textit{integer} and \textit{numeric}. \textit{numeric} can be an
  instance of \gmpz, \textit{Fixnum}, \gmpq, \gmpf, or \textit{Bignum}.
}
\end{tabular}
\newline\newline

\begin{tabular}{p{\methwidth} l r}
\toprule
\textbf{add!} & & \textit{integer}.add!(\textit{numeric}) $\rightarrow$
\textit{numeric} \\
\cmidrule(r){2-3}
& \multicolumn{2}{p{\defnwidth}}{
  Sums \textit{integer} and \textit{numeric}, in place. \textit{numeric} can be an
  instance of \gmpz, \textit{Fixnum}, \gmpq, \gmpf, or \textit{Bignum}.
}
\end{tabular}
\newline\newline

\begin{tabular}{p{\methwidth} l r}
\toprule
\textbf{-} & & \textit{integer} - \textit{numeric} $\rightarrow$ \textit{numeric} \\
& & \textit{integer}.sub!(\textit{numeric}) $\rightarrow$ \textit{numeric} \\
\cmidrule(r){2-3}
& \multicolumn{2}{p{\defnwidth}}{
  Returns the difference of \textit{integer} and \textit{numeric}. The destructive method
  calculates the difference in place. \textit{numeric} can be an instance of
  \gmpz, \textit{Fixnum}, \gmpq, \gmpf, or \textit{Bignum}. Here are some
  examples:\newline
  
  \texttt{seven     = GMP::Z(7) \newline
          nine \    = GMP::Z(9) \newline
          half \    = GMP::Q(1,2) \newline
          pi \quad\ = GMP::F("3.14") \newline
          nine - 5 \qquad\quad \#=> 4 (GMP Integer) \newline
          nine - seven \quad \#=> 2 (GMP Integer) \newline
          nine - (2**32) \#=> -4294967287 (GMP Integer) \newline
          nine - nine \quad\ \#=> 0 (GMP Integer) \newline
          nine - half \quad\ \#=> 8.5 (GMP Rational) \newline
          nine - pi \qquad\  \#=> 5.86 (GMP Float)}
  
}
\end{tabular}
\newline\newline

\begin{tabular}{p{\methwidth} l r}
\toprule
\textbf{*} & & \textit{integer} * \textit{numeric} $\rightarrow$ \textit{numeric} \\
& & \textit{integer}.mul(\textit{numeric}) $\rightarrow$ \textit{numeric} \\
& & \textit{integer}.mul!(\textit{numeric}) $\rightarrow$ \textit{numeric} \\
\cmidrule(r){2-3}
& \multicolumn{2}{p{\defnwidth}}{
  Returns the product of \textit{integer} and \textit{numeric}. The destructive method
  calculates the product in place. \textit{numeric} can be an instance of
  \gmpz, \textit{Fixnum}, \gmpq, \gmpf, or \textit{Bignum}.
}
\end{tabular}
\newline\newline

\begin{tabular}{p{\methwidth} l r}
\toprule
\textbf{addmul!} & & \textit{integer}.addmul!(\textit{b}, \textit{c}) $\rightarrow$ \textit{numeric} \\
\cmidrule(r){2-3}
& \multicolumn{2}{p{\defnwidth}}{
  Sets \textit{integer} to the sum of \textit{integer} and the product of \textit{b} and
  \textit{c}. This destructive method calculates the result in place. Both \textit{b} and
  \textit{c} can be an instance of \gmpz, \textit{Fixnum}, or \textit{Bignum}.
}
\end{tabular}
\newline\newline

\begin{tabular}{p{\methwidth} l r}
\toprule
\textbf{submul!} & & \textit{integer}.submul!(\textit{b}, \textit{c}) $\rightarrow$ \textit{numeric} \\
\cmidrule(r){2-3}
& \multicolumn{2}{p{\defnwidth}}{
  Sets \textit{integer} to the difference of \textit{integer} and the product of \textit{b} and
  \textit{c}. This destructive method calculates the result in place. Both \textit{b} and
  \textit{c} can be an instance of \gmpz, \textit{Fixnum}, or \textit{Bignum}.
}
\end{tabular}
\newline\newline

\begin{tabular}{p{\methwidth} l r}
\toprule
\textbf{\textless\textless} & & \textit{integer} \textless\textless \textit{numeric} $\rightarrow$ \textit{integer} \\
\cmidrule(r){2-3}
& \multicolumn{2}{p{\defnwidth}}{
  Returns \textit{integer} times 2 to the \textit{numeric} power. This can also be
  defined as a left shift by \textit{numeric} bits.
}
\end{tabular}
\newline\newline

\begin{tabular}{p{\methwidth} l r}
\toprule
\textbf{-@} & & -\textit{integer}\\
& & \textit{integer}.neg\\
& & \textit{integer}.neg!\\
\cmidrule(r){2-3}
& \multicolumn{2}{p{\defnwidth}}{
  Returns the negation, the additive inverse, of \textit{integer}. The destructive method
  negates in place.
}
\end{tabular}
\newline\newline

\begin{tabular}{p{\methwidth} l r}
\toprule
\textbf{abs} & & \textit{integer}.abs\\
& & \textit{integer}.abs!\\
\cmidrule(r){2-3}
& \multicolumn{2}{p{\defnwidth}}{
  Returns the absolute value of \textit{integer}. The destructive method calculates the
  absolute value in place.
}
\end{tabular}

\subsection{Integer Division}

\begin{tabular}{p{\methwidth} l r}
\toprule
\textbf{tdiv} & & $integer$.tdiv($numeric$) $\rightarrow$ $integer$\\
\cmidrule(r){2-3}
& \multicolumn{2}{p{\defnwidth}}{
  Returns the division of $integer$ by $numeric$, truncated. $numeric$ can be an instance
  of \gmpz, $Fixnum$, $Bignum$. The return object's class is always \gmpz.
}
\end{tabular}
\newline\newline

\begin{tabular}{p{\methwidth} l r}
\toprule
\textbf{fdiv} & & $integer$.fdiv($numeric$) $\rightarrow$ $integer$\\
\cmidrule(r){2-3}
& \multicolumn{2}{p{\defnwidth}}{
  Returns the division of $integer$ by $numeric$, floored. $numeric$ can be an instance
  of \gmpz, $Fixnum$, $Bignum$. The return object's class is always \gmpz.
}
\end{tabular}
\newline\newline

\begin{tabular}{p{\methwidth} l r}
\toprule
\textbf{cdiv} & & $integer$.cdiv($numeric$) $\rightarrow$ $integer$\\
\cmidrule(r){2-3}
& \multicolumn{2}{p{\defnwidth}}{
  Returns the ceiling division of $integer$ by $numeric$. $numeric$ can be an instance of
  \gmpz, $Fixnum$, $Bignum$. The return object's class is always \gmpz.
}
\end{tabular}
\newline\newline

\begin{tabular}{p{\methwidth} l r}
\toprule
\textbf{tmod} & & $integer$.tmod($numeric$) $\rightarrow$ $integer$\\
\cmidrule(r){2-3}
& \multicolumn{2}{p{\defnwidth}}{
  Returns the remainder after truncated division of $integer$ by $numeric$. $numeric$ can
  be an instance of \gmpz, $Fixnum$, or $Bignum$. The return object's class is always
  \gmpz.
}
\end{tabular}
\newline\newline

\begin{tabular}{p{\methwidth} l r}
\toprule
\textbf{fmod} & & $integer$.fmod($numeric$) $\rightarrow$ $integer$\\
\cmidrule(r){2-3}
& \multicolumn{2}{p{\defnwidth}}{
  Returns the remainder after floored division of $integer$ by $numeric$. $numeric$ can
  be an instance of \gmpz, $Fixnum$, or $Bignum$. The return object's class is always
  \gmpz.
}
\end{tabular}
\newline\newline

\begin{tabular}{p{\methwidth} l r}
\toprule
\textbf{cmod} & & \textit{integer}.cmod(\textit{numeric}) $\rightarrow$ \textit{integer}\\
\cmidrule(r){2-3}
& \multicolumn{2}{p{\defnwidth}}{
  Returns the remainder after ceilinged division of $integer$ by $numeric$. $numeric$ can
  be an instance of \gmpz, $Fixnum$, or $Bignum$. The return object's class is always
  \gmpz.
}
\end{tabular}
\newline\newline

\begin{tabular}{p{\methwidth} l r}
\toprule
\textbf{\%} & & \textit{integer} \% \textit{numeric} $\rightarrow$ \textit{integer}\\
\cmidrule(r){2-3}
& \multicolumn{2}{p{\defnwidth}}{
  Returns $integer$ modulo $numeric$. $numeric$ can be an instance of \gmpz, $Fixnum$,
  or $Bignum$. The return object's class is always \gmpz.
}
\end{tabular}
\newline\newline

\begin{tabular}{p{\methwidth} l r}
\toprule
\textbf{divisible?} & & \textit{integer}.divisible? \textit{numeric} $\rightarrow$ \textit{boolean}\\
\cmidrule(r){2-3}
& \multicolumn{2}{p{\defnwidth}}{
  Returns whether $integer$ is divisible by $numeric$. $numeric$ can be an instance of \gmpz, $Fixnum$,
  or $Bignum$.
}
\end{tabular}

\newpage
\subsection{Integer Exponentiation}

\begin{tabular}{p{\methwidth} l r}
\toprule
\textbf{**} & & \textit{integer} ** \textit{numeric} $\rightarrow$ \textit{numeric} \\
& & \textit{integer}.pow(\textit{numeric}) $\rightarrow$ \textit{numeric} \\
& & GMP::Z.pow(\textit{integer}, \textit{numeric}) $\rightarrow$ \textit{numeric} \\
\cmidrule(r){2-3}
& \multicolumn{2}{p{\defnwidth}}{
  Returns $integer$ raised to the $numeric$ power. In the singleton method (\textit{GMP::Z.pow()}),
  \textit{integer} can be either a \gmpz, $Fixnum$, $Bignum$, or $String$.
}
\end{tabular}
\newline\newline

\begin{tabular}{p{\methwidth} l r}
\toprule
\textbf{powmod} & & \textit{integer}.powmod(\textit{exp}, \textit{mod}) $\rightarrow$ \textit{integer} \\
\cmidrule(r){2-3}
& \multicolumn{2}{p{\defnwidth}}{
  Returns $integer$ raised to the $exp$ power, modulo $mod$. Negative $exp$ is supported
  if an inverse, $integer^{-1}$ modulo $mod$, exists. If an inverse doesn't exist then a
  divide by zero exception is raised.
}
\end{tabular}

\subsection{Integer Roots}

\begin{tabular}{p{\methwidth} l r}
\toprule
\textbf{root} & & \textit{integer}.root(\textit{numeric}) $\rightarrow$ \textit{numeric} \\
\cmidrule(r){2-3}
& \multicolumn{2}{p{\defnwidth}}{
  Returns the integer part of the $numeric$'th root of $integer$.
}
\end{tabular}
\newline\newline

\begin{tabular}{p{\methwidth} l r}
\toprule
\textbf{sqrt} & & \textit{integer}.sqrt $\rightarrow$ \textit{numeric} \\
& & \textit{integer}.sqrt! $\rightarrow$ \textit{numeric} \\
\cmidrule(r){2-3}
& \multicolumn{2}{p{\defnwidth}}{
  Returns the truncated integer part of the square root of $integer$.
}
\end{tabular}
\newline\newline

\begin{tabular}{p{\methwidth} l r}
\toprule
\textbf{sqrtrem} & & \textit{integer}.sqrtrem $\rightarrow$ \textit{sqrt}, \textit{rem} \\
\cmidrule(r){2-3}
& \multicolumn{2}{p{\defnwidth}}{
  Returns the truncated integer part of the square root of $integer$ as $sqrt$ and the
  remainder, $integer - sqrt * sqrt$, as $rem$, which will be zero if $integer$ is a
  perfect square.
}
\end{tabular}
\newline\newline

\begin{tabular}{p{\methwidth} l r}
\toprule
\textbf{power?} & & \textit{integer}.power? $\rightarrow$ \textit{true} \textbar\  \textit{false} \\
\cmidrule(r){2-3}
& \multicolumn{2}{p{\defnwidth}}{
  Returns true if $integer$ is a perfect power, i.e., if there exist integers $a$ and
  $b$, with $b > 1$, such that $integer$ equals $a$ raised to the power $b$.
  \newline\newline
  Under this definition both 0 and 1 are considered to be perfect powers. Negative values
  of integers are accepted, but of course can only be odd perfect powers.
}
\end{tabular}
\newline\newline

\begin{tabular}{p{\methwidth} l r}
\toprule
\textbf{square?} & & \textit{integer}.square? $\rightarrow$ \textit{true} \textbar\  \textit{false} \\
\cmidrule(r){2-3}
& \multicolumn{2}{p{\defnwidth}}{
  Returns true if $integer$ is a perfect square, i.e., if the square root of
  $integer$ is an integer. Under this definition both 0 and 1 are considered to be
  perfect squares.
}
\end{tabular}

\subsection{Number Theoretic Functions}

\begin{tabular}{p{\methwidth} l r}
\toprule
\textbf{probab\_prime?} & & $integer$.probab\_prime?($reps = 5$) $\rightarrow$ 0, 1, or 2 \\
\cmidrule(r){2-3}
& \multicolumn{2}{p{\defnwidth}}{
  Determine whether $integer$ is prime. Returns 2 if $integer$ is definitely prime,
  returns 1 if $integer$ is probably prime (without being certain), or returns 0 if
  $integer$ is definitely composite. 
  \newline\newline
  This function does some trial divisions, then some Miller-Rabin probabilistic primality
  tests. $reps$ controls how many such tests are done, 5 to 10 is a reasonable number,
  more will reduce the chances of a composite being returned as �probably prime�.
  \newline\newline
  Miller-Rabin and similar tests can be more properly called compositeness tests. Numbers
  which fail are known to be composite but those which pass might be prime or might be
  composite. Only a few composites pass, hence those which pass are considered probably
  prime.
}
\end{tabular}
\newline\newline

\begin{tabular}{p{\methwidth} l r}
\toprule
\textbf{next\_prime} & & $integer$.next\_prime $\rightarrow$ $prime$ \\
& & $integer$.nextprime $\rightarrow$ $prime$ \\
& & $integer$.next\_prime! $\rightarrow$ $prime$ \\
& & $integer$.nextprime! $\rightarrow$ $prime$ \\
\cmidrule(r){2-3}
& \multicolumn{2}{p{\defnwidth}}{
  Returns the next prime greater than $integer$. The destructive method sets $integer$ to
  the next prime greater than $integer$.
  \newline\newline
  This function uses a probabilistic algorithm to identify primes. For practical purposes
  it's adequate, the chance of a composite passing will be extremely small.
}
\end{tabular}
\newline\newline

\begin{tabular}{p{\methwidth} l r}
\toprule
\textbf{gcd} & & $a$.gcd($b$) $\rightarrow$ $g$ \\
\cmidrule(r){2-3}
& \multicolumn{2}{p{\defnwidth}}{
  Computes the greatest common divisor of $a$ and $b$. $g$ will always be positive, even
  if $a$ or $b$ is negative. $b$ can be an instance of \gmpz, $Fixnum$, or $Bignum$.
  \newline\newline
  \texttt{GMP::Z(24).gcd(GMP::Z(8)) \quad \#=> GMP::Z(8) \newline
          GMP::Z(24).gcd(8) \quad \#=> GMP::Z(8) \newline
          GMP::Z(24).gcd(2**32) \quad \#=> GMP::Z(8)}
}
\end{tabular}
\newline\newline

\begin{tabular}{p{\methwidth} l r}
\toprule
\textbf{gcdext} & & $a$.gcd($b$) $\rightarrow$ $g$, $s$, $t$ \\
\cmidrule(r){2-3}
& \multicolumn{2}{p{\defnwidth}}{
  Computes the greatest common divisor of $a$ and $b$, in addition to $s$ and $t$, the
  coefficients satisfying $a*s + b*t = g$. $g$ will always be positive, even if $a$ or
  $b$ is negative. $s$ and $t$ are chosen such that $|s| <= |b|$ and $|t| <= |a|$. $b$
  can be an instance of \gmpz, $Fixnum$, or $Bignum$.
}
\end{tabular}
\newline\newline

\begin{tabular}{p{\methwidth} l r}
\toprule
\textbf{invert} & & $a$.invert($m$) $\rightarrow$ $integer$ \\
\cmidrule(r){2-3}
& \multicolumn{2}{p{\defnwidth}}{
  Computes the inverse of $a$ mod $m$. $m$ can be an instance of \gmpz, $Fixnum$, or
  $Bignum$.
  \newline\newline
  \texttt{GMP::Z(2).invert(GMP::Z(11)) \quad \#=> GMP::Z(6) \newline
          GMP::Z(3).invert(11) \quad \#=> GMP::Z(4) \newline
          GMP::Z(5).invert(11) \quad \#=> GMP::Z(9)}
}
\end{tabular}
\newline\newline

\begin{tabular}{p{\methwidth} l r}
\toprule
\textbf{jacobi} & & $a$.jacobi($b$) $\rightarrow$ $integer$ \\
& & GMP::Z.jacobi($a$, $b$) $\rightarrow$ $integer$ \\
\cmidrule(r){2-3}
& \multicolumn{2}{p{\defnwidth}}{
  Returns the Jacobi symbol $(a/b)$. This is defined only for $b$ odd. If $b$ is even, a
  range exception will be raised.
  \newline\newline
  \textit{GMP::Z.jacobi} (the instance method) requires $b$ to be an instance of \gmpz.
  \newline
  \textit{GMP::Z\#jacobi} (the class method) requires $a$ and $b$ each to be an instance of
  \gmpz, $Fixnum$, or $Bignum$.
}
\end{tabular}
\newline\newline

\begin{tabular}{p{\methwidth} l r}
\toprule
\textbf{legendre} & & $a$.legendre($b$) $\rightarrow$ $integer$ \\
\cmidrule(r){2-3}
& \multicolumn{2}{p{\defnwidth}}{
  Returns the Legendre symbol $(a/b)$. This is defined only for $p$ an odd positive
  prime. If $p$ is even, negative, or composite, a range exception will be raised.
}
\end{tabular}
\newline\newline

\begin{tabular}{p{\methwidth} l r}
\toprule
\textbf{remove} & & $n$.remove($factor$) $\rightarrow$ ($integer$, $times$) \\
\cmidrule(r){2-3}
& \multicolumn{2}{p{\defnwidth}}{
  Remove all occurrences of the factor $factor$ from $n$. $factor$ can be an instance of
  \gmpz, $Fixnum$, or $Bignum$. $integer$ is the resulting integer, an instance of
  \gmpz. $times$ is how many times $factor$ was removed, a $Fixnum$.
}
\end{tabular}
\newline\newline

\begin{tabular}{p{\methwidth} l r}
\toprule
\textbf{fac} & & GMP::Z.fac($n$) $\rightarrow$ $integer$ \\
\cmidrule(r){2-3}
& \multicolumn{2}{p{\defnwidth}}{
  Returns $n!$, or, $n$ factorial.
}
\end{tabular}
\newline\newline

\begin{tabular}{p{\methwidth} l r}
\toprule
\textbf{fib} & & GMP::Z.fib($n$) $\rightarrow$ $integer$ \\
\cmidrule(r){2-3}
& \multicolumn{2}{p{\defnwidth}}{
  Returns $F[n]$, the $n$th Fibonacci number.
}
\end{tabular}
\newline\newline

\begin{tabular}{p{\methwidth} l r}
\toprule
\textbf{fib2} & & GMP::Z.fib2($n$) $\rightarrow$ $integer$ \\
\cmidrule(r){2-3}
& \multicolumn{2}{p{\defnwidth}}{
  Returns $F[n]$ and $F[n-1]$, the $n$th and $n-1$th Fibonacci numbers.
}
\end{tabular}

\subsection{Integer Comparisons}

\begin{tabular}{p{\methwidth} l r}
\toprule
\textbf{\textless=\textgreater} & & $a$ \textless=\textgreater\ $b$ $\rightarrow$ $fixnum$ \\
\cmidrule(r){2-3}
& \multicolumn{2}{p{\defnwidth}}{
  Returns a negative Fixnum if $a$ is less than $b$.\newline
  Returns 0 if $a$ is equal to $b$.\newline
  Returns a positive Fixnum if $a$ is greater than $b$.
}
\end{tabular}
\newline\newline

\begin{tabular}{p{\methwidth} l r}
\toprule
\textbf{\textless} & & $a$ \textless\ $b$ $\rightarrow$ $boolean$ \\
\cmidrule(r){2-3}
& \multicolumn{2}{p{\defnwidth}}{
  Returns true if $a$ is less than $b$.
}
\end{tabular}
\newline\newline

\begin{tabular}{p{\methwidth} l r}
\toprule
\textbf{\textless=} & & $a$ \textless=\ $b$ $\rightarrow$ $boolean$ \\
\cmidrule(r){2-3}
& \multicolumn{2}{p{\defnwidth}}{
  Returns true if $a$ is less than or equal to $b$.
}
\end{tabular}
\newline\newline

\begin{tabular}{p{\methwidth} l r}
\toprule
\textbf{==} & & $a$ == $b$ $\rightarrow$ $boolean$ \\
\cmidrule(r){2-3}
& \multicolumn{2}{p{\defnwidth}}{
  Returns true if $a$ is equal to $b$.
}
\end{tabular}
\newline\newline

\begin{tabular}{p{\methwidth} l r}
\toprule
\textbf{\textgreater=} & & $a$ \textgreater= $b$ $\rightarrow$ $boolean$ \\
\cmidrule(r){2-3}
& \multicolumn{2}{p{\defnwidth}}{
  Returns true if $a$ is greater than or equal to $b$.
}
\end{tabular}
\newline\newline

\begin{tabular}{p{\methwidth} l r}
\toprule
\textbf{\textgreater} & & $a$ \textgreater\ $b$ $\rightarrow$ $boolean$ \\
\cmidrule(r){2-3}
& \multicolumn{2}{p{\defnwidth}}{
  Returns true if $a$ is greater than $b$.
}
\end{tabular}
\newline\newline

\begin{tabular}{p{\methwidth} l r}
\toprule
\textbf{cmpabs} & & $a$.cmpabs($b$) $\rightarrow$ $fixnum$ \\
\cmidrule(r){2-3}
& \multicolumn{2}{p{\defnwidth}}{
  Returns a negative Fixnum if abs($a$) is less than abs($b$).\newline
  Returns 0 if abs($a$) is equal to abs($b$).\newline
  Returns a positive Fixnum if abs($a$) is greater than abs($b$).
}
\end{tabular}
\newline\newline

\begin{tabular}{p{\methwidth} l r}
\toprule
\textbf{sgn} & & $a$.sgn $\rightarrow$ $-1$, $0$, or $1$ \\
\cmidrule(r){2-3}
& \multicolumn{2}{p{\defnwidth}}{
  Returns -1 if $a$ is less than $b$.\newline
  Returns 0 if $a$ is equal to $b$.\newline
  Returns 1 if $a$ is greater than $b$.
}
\end{tabular}
\newline\newline

\begin{tabular}{p{\methwidth} l r}
\toprule
\textbf{eql?} & & $a$.eql?($b$) $\rightarrow$ $boolean$ \\
\cmidrule(r){2-3}
& \multicolumn{2}{p{\defnwidth}}{
  Used when comparing objects as Hash keys.
}
\end{tabular}
\newline\newline

\begin{tabular}{p{\methwidth} l r}
\toprule
\textbf{hash} & & $a$.hash $\rightarrow$ $string$ \\
\cmidrule(r){2-3}
& \multicolumn{2}{p{\defnwidth}}{
  Used when comparing objects as Hash keys.
}
\end{tabular}

\subsection{Integer Logic and Bit Fiddling}

\begin{tabular}{p{\methwidth} l r}
\toprule
\textbf{and} & & $a$ \& $b$ $\rightarrow$ $integer$ \\
\cmidrule(r){2-3}
& \multicolumn{2}{p{\defnwidth}}{
  Returns $integer$, the bitwise and of $a$ and $b$.
}
\end{tabular}
\newline\newline

\begin{tabular}{p{\methwidth} l r}
\toprule
\textbf{ior} & & $a$ \textbar\ $b$ $\rightarrow$ $integer$ \\
\cmidrule(r){2-3}
& \multicolumn{2}{p{\defnwidth}}{
  Returns $integer$, the bitwise inclusive or of $a$ and $b$.
}
\end{tabular}
\newline\newline

\begin{tabular}{p{\methwidth} l r}
\toprule
\textbf{xor} & & $a$ \textasciicircum\ $b$ $\rightarrow$ $integer$ \\
\cmidrule(r){2-3}
& \multicolumn{2}{p{\defnwidth}}{
  Returns $integer$, the bitwise exclusive or of $a$ and $b$.
}
\end{tabular}
\newline\newline

\begin{tabular}{p{\methwidth} l r}
\toprule
\textbf{com} & & $integer$.com $\rightarrow$ $complement$ \\
& & $integer$.com! $\rightarrow$ $complement$ \\
\cmidrule(r){2-3}
& \multicolumn{2}{p{\defnwidth}}{
  Returns the one's complement of $integer$. The destructive method sets $integer$ to
  the one's complement of $integer$.
}
\end{tabular}
\newline\newline

\begin{tabular}{p{\methwidth} l r}
\toprule
\textbf{popcount} & & $n$.popcount $\rightarrow$ $fixnum$ \\
\cmidrule(r){2-3}
& \multicolumn{2}{p{\defnwidth}}{
  If $n>=0$, return the population count of $n$, which is the number of 1 bits in the
  binary representation. If $n<0$, the number of 1s is infinite, and the return value
  is the largest possible $mp\_bitcnt\_t$. 
}
\end{tabular}
\newline\newline

\begin{tabular}{p{\methwidth} l r}
\toprule
\textbf{scan0} & & $n$.scan0($i$) $\rightarrow$ $integer$ \\
\cmidrule(r){2-3}
& \multicolumn{2}{p{\defnwidth}}{
  Scans $n$, starting from bit $i$, towards more significant bits, until the first 0 bit
  is found. Return the index of the found bit.
  \newline\newline
  If the bit at $i$ is already what's sought, then $i$ is returned.
  \newline\newline
  If there's no bit found, then $INT2FIX(ULONG\_MAX)$ is returned. This will happen in
  scan0 past the end of a negative number.
}
\end{tabular}
\newline\newline

\begin{tabular}{p{\methwidth} l r}
\toprule
\textbf{scan1} & & $n$.scan1($i$) $\rightarrow$ $integer$ \\
\cmidrule(r){2-3}
& \multicolumn{2}{p{\defnwidth}}{
  Scans $n$, starting from bit $i$, towards more significant bits, until the first 1 bit
  is found. Return the index of the found bit.
  \newline\newline
  If the bit at $i$ is already what's sought, then $i$ is returned.
  \newline\newline
  If there's no bit found, then $INT2FIX(ULONG\_MAX)$ is returned. This will happen in
  scan1 past the end of a negative number.
}
\end{tabular}
\newline\newline

\begin{tabular}{p{\methwidth} l r}
\toprule
\textbf{[]} & & $n$[$bit\_index$] $\rightarrow$ $0$ or $1$ \\
\cmidrule(r){2-3}
& \multicolumn{2}{p{\defnwidth}}{
  Tests bit $bit\_index$ in $n$ and return $0$ or $1$ accordingly.
}
\end{tabular}
\newline\newline

\begin{tabular}{p{\methwidth} l r}
\toprule
\textbf{[]=} & & $n$[$bit\_index$]=$i$ $\rightarrow$ $nil$ \\
\cmidrule(r){2-3}
& \multicolumn{2}{p{\defnwidth}}{
  Sets bit $bit\_index$ in $n$ to $i$.
}
\end{tabular}

\newpage
\subsection{Miscellaneous Integer Functions}

\begin{tabular}{p{\methwidth} l r}
\toprule
\textbf{odd?} & & $n$.odd? $\rightarrow$ $boolean$ \\
\cmidrule(r){2-3}
& \multicolumn{2}{p{\defnwidth}}{
  Returns whether $n$ is odd.
}
\end{tabular}
\newline\newline

\begin{tabular}{p{\methwidth} l r}
\toprule
\textbf{even?} & & $n$.even? $\rightarrow$ $boolean$ \\
\cmidrule(r){2-3}
& \multicolumn{2}{p{\defnwidth}}{
  Returns whether $n$ is even.
}
\end{tabular}
\newline\newline

\begin{tabular}{p{\methwidth} l r}
\toprule
\textbf{sizeinbase} & & $n$.sizeinbase($b$) $\rightarrow$ $digits$ \\
\cmidrule(r){2-3}
& \multicolumn{2}{p{\defnwidth}}{
  Returns the number of digits in base $b$. $b$ can vary between 2 and 62.
}
\end{tabular}
\newline\newline

\begin{tabular}{p{\methwidth} l r}
\toprule
\textbf{size\_in\_bin} & & $n$.size\_in\_bin $\rightarrow$ $digits$ \\
\cmidrule(r){2-3}
& \multicolumn{2}{p{\defnwidth}}{
  Returns the number of digits in $n$'s binary representation.
}
\end{tabular}

\subsection{Integer Special Functions}

\begin{tabular}{p{\methwidth} l r}
\toprule
\textbf{size} & & $integer$.size $\rightarrow$ $fixnum$ \\
\cmidrule(r){2-3}
& \multicolumn{2}{p{\defnwidth}}{
  Returns the size of \textit{integer} measured in number of limbs. If \textit{integer}
  is zero, then the returned value will be zero.
}
\end{tabular}

\newpage
\section{Rational Functions}

\subsection{Initializing, Assigning Rationals}

\begin{tabular}{p{\methwidth} l r}
\toprule
\textbf{new} & & GMP::Q.new $\rightarrow$ \textit{rational} \\
& & GMP::Q.new(\textit{numerator = 0}, \textit{denominator = 1}) $\rightarrow$ \textit{rational} \\
& & GMP::Q.new(\textit{str}) $\rightarrow$ \textit{rational} \\
\cmidrule(r){2-3}
& \multicolumn{2}{p{\defnwidth}}{
  This method creates a new \gmpq rational number. It takes two optional
  arguments for the value of the numerator and denominator. These arguments can each be
  an instance of several classes. Here are some
  examples:\newline

  \texttt{GMP::Q.new \qqqquad\qqqquad \#=> 0 (default) \newline
          GMP::Q.new(1) \qqqquad\qquad\  \#=> 1 (Ruby Fixnum) \newline
          GMP::Q.new(1,3) \qqqquad\quad\  \#=> 1/3 (Ruby Fixnums) \newline
          GMP::Q.new("127") \qqqquad\  \#=> 127 (Ruby String)\newline
          GMP::Q.new(4294967296) \qquad \#=> 4294967296 (Ruby Bignum)\newline
          GMP::Q.new(GMP::Z.new(31))  \#=> 31 (GMP Integer)}
}
\end{tabular}
\newline\newline

There is also a convenience method available, \texttt{GMP::Q()}.\\

\subsection{Converting Rationals}

\begin{tabular}{p{\methwidth} l r}
\toprule
\textbf{to\_d} & & \textit{rational}.to\_d $\rightarrow$ \textit{float} \\
\cmidrule(r){2-3}
& \multicolumn{2}{p{\defnwidth}}{
  Returns \textit{rational} as an Float if \textit{rational} fits in a Float.

  Otherwise returns the least significant part of \texttt{rational}, with the same sign as
  \textit{rational}.

  If \textit{rational} is too big to fit in a Float, the returned result is probably not
  very useful.
}
\end{tabular}
\newline\newline

\begin{tabular}{p{\methwidth} l r}
\toprule
\textbf{to\_s} & & \textit{rational}.to\_s $\rightarrow$ \textit{str} \\
\cmidrule(r){2-3}
& \multicolumn{2}{p{\defnwidth}}{
  Converts \textit{rational} to a string.
}
\end{tabular}
\newline\newline

\subsection{Rational Arithmetic}

\begin{tabular}{p{\methwidth} l r}
\toprule
\textbf{+} & & \textit{rational} + \text{numeric} $\rightarrow$ \textit{numeric} \\
\end{tabular}
\newline\newline

\newpage
\section{Floating-point Functions}

\subsection{Initializing, Assigning Floats}

\begin{tabular}{p{\methwidth} l r}
\toprule
\textbf{new} & & GMP::F.new $\rightarrow$ \textit{float} \\
& & GMP::F.new(\textit{numeric}, \begin{small}precision = default, rnd\_mode = GMP\_RNDN\end{small}) $\rightarrow$ \textit{float} \\
& & GMP::F.new(\textit{str}, \begin{small}base = 0\end{small}) $\rightarrow$ \textit{float} \\
\cmidrule(r){2-3}
& \multicolumn{2}{p{\defnwidth}}{
  This method creates a new \gmpfs float. It typically takes one optional
  argument for the value of the float. This argument can be one of several
  classes. Optionally, a precision can be passed.\newline

  If MPFR is available, an optional rounding mode can also be passed.
  \newline

  If the first argument is a String, then a second argument, the base, may be
  optionally supplied.  Here are some examples:\newline

  \texttt{GMP::F.new \qqqquad\qqqquad \#=> 0 (default) \newline
          GMP::F.new(5) \qqqquad\qquad\  \#=> 5 (Ruby Fixnum) \newline
          GMP::F.new(GMP::Z.new(31))  \#=> 31 (GMP Integer)\newline
          GMP::F.new(3**41) \qqqquad\  \#=> 0.36472996377170788e+20\newline
          >\qqqquad\qqqquad\qqqquad\qquad\quad (Ruby Bignum)\newline
          GMP::F.new(3**41, 32) \qquad\  \#=> 0.36472996375+20\newline
          >\qqqquad\qqqquad\qqqquad\qquad\quad (Ruby Bignum with precision)\newline
          GMP::F.new(3**41, 32, GMP::GMP\_RNDU) \#=> 0.36472996375+20\newline
          >\qqqquad\qqqquad\  (Ruby Bignum with precision and a rounding mode)\newline
          GMP::F.new("20") \qqqquad\quad \#=> 20 (Ruby String)\newline
          GMP::F.new("0x20") \qqqquad \#=> 32 (Ruby hexadecimal-format String)\newline
          GMP::F.new("111", 16) \qquad\  \#=> 111 (Ruby String with precision)\newline
          GMP::F.new("111", 16, 2) \quad \#=> 7\newline
          >\qqqquad\qqqquad\qqqquad\quad (Ruby String with precision and a base)}
}
\end{tabular}
\newline\newline

There is also a convenience method available, \texttt{GMP::F()}.\\

\begin{tabular}{p{\methwidth} l r}
\toprule
\textbf{nan} & (MPFR only) & GMP::F.nan $\rightarrow$ \textit{NaN} \\
\cmidrule(r){2-3}
& \multicolumn{2}{p{\defnwidth}}{
  Returns NaN, an instance of \gmpfs.
}
\end{tabular}
\newline\newline

\begin{tabular}{p{\methwidth} l r}
\toprule
\textbf{inf} & (MPFR only) & GMP::F.inf(\textit{sign} = 1) $\rightarrow$ \textit{Inf} \\
\cmidrule(r){2-3}
& \multicolumn{2}{p{\defnwidth}}{
  Returns Inf (positive infinity) or -Inf (negative infinity), an instance of
  \gmpfs, based on the sign of \textit{sign}, which must be a Fixnum, and
  defaults to 1.
}
\end{tabular}
\newline\newline

\begin{tabular}{p{\methwidth} l r}
\toprule
\textbf{zero} & (MPFR only) & GMP::F.zero(\textit{sign} = 1) $\rightarrow$ \textit{zero} \\
\cmidrule(r){2-3}
& \multicolumn{2}{p{\defnwidth}}{
  Returns zero or -zero, an instance of \gmpfs, based on the sign of
  \textit{sign}, which must be a Fixnum, and defaults to 1.
}
\end{tabular}
\newline\newline

\subsection{Floating-point Conversion Functions}

Every method below accepts two additional parameters in addition to any
required parameters. These are $rnd\_mode$, the rounding mode to use in
calculation, which defaults to \textit{GMP::GMP\_RNDN}, and $res\_prec$, the
precision of the result, which defaults to the \textit{f.prec}, the precision
of $f$.\\

TONS OF MISSING DOCUMENTATION\\

\begin{tabular}{p{\methwidth} l r}
\toprule
\textbf{frexp}   & (MPFR 3.1 only) & $f$.frexp(\begin{small}rnd\_mode = GMP\_RNDN, res\_prec=$f$.prec\end{small}) $\rightarrow$ $exp$, $g$ \\
\cmidrule(r){2-3}
& \multicolumn{2}{p{\defnwidth}}{
  Set $exp$ and $y$ such that $0.5 <= abs(y) < 1$ and $y$ times 2 raised to
  $exp$ equals $x$ rounded to $res\_prec$, using $rnd\_mode$. If $x$ is zero,
  then $y$ is set to a zero of the same sign and $exp$ is set to 0. If $x$ is
  NaN or an infinity, then $y$ is set to the same value and $exp$ is undefined.
}
\end{tabular}
\newline\newline

\subsection{Floating-point Special Functions (MPFR Only)}

Every method below accepts two additional parameters in addition to any
required parameters. These are $rnd\_mode$, the rounding mode to use in
calculation, which defaults to \textit{GMP::GMP\_RNDN}, and $res\_prec$, the
precision of the result, which defaults to the \textit{f.prec}, the precision
of $f$.\\

\begin{tabular}{p{\methwidth} l r}
\toprule
\textbf{log}   & & $f$.log(\begin{small}rnd\_mode = GMP\_RNDN, res\_prec=$f$.prec\end{small}) $\rightarrow$ $g$ \\
\textbf{log2}  & & $f$.log2(\begin{small}rnd\_mode = GMP\_RNDN, res\_prec=$f$.prec\end{small}) $\rightarrow$ $g$ \\
\textbf{log10} & & $f$.log10(\begin{small}rnd\_mode = GMP\_RNDN, res\_prec=$f$.prec\end{small}) $\rightarrow$ $g$ \\
\cmidrule(r){2-3}
& \multicolumn{2}{p{\defnwidth}}{
  Returns the natural log, $\log_2$, and $\log_10$ of $f$, respectively. Returns $-Inf$ if $f$ is $-0$.
}
\end{tabular}
\newline\newline

\begin{tabular}{p{\methwidth} l r}
\toprule
\textbf{exp}   & & $f$.exp(\begin{small}rnd\_mode = GMP\_RNDN, res\_prec=$f$.prec\end{small}) $\rightarrow$ $g$ \\
\textbf{exp2}  & & $f$.exp2(\begin{small}rnd\_mode = GMP\_RNDN, res\_prec=$f$.prec\end{small}) $\rightarrow$ $g$ \\
\textbf{exp10} & & $f$.exp10(\begin{small}rnd\_mode = GMP\_RNDN, res\_prec=$f$.prec\end{small}) $\rightarrow$ $g$ \\
\cmidrule(r){2-3}
& \multicolumn{2}{p{\defnwidth}}{
  Returns the exponential of $f$, $2$ to the power of $f$, and $10$ to the power of $f$, respectively.
}
\end{tabular}
\newline\newline

\begin{tabular}{p{\methwidth} l r}
\toprule
\textbf{cos} & & $f$.cos(\begin{small}rnd\_mode = GMP\_RNDN, res\_prec=$f$.prec\end{small}) $\rightarrow$ $g$ \\
\textbf{sin} & & $f$.sin(\begin{small}rnd\_mode = GMP\_RNDN, res\_prec=$f$.prec\end{small}) $\rightarrow$ $g$ \\
\textbf{tan} & & $f$.tan(\begin{small}rnd\_mode = GMP\_RNDN, res\_prec=$f$.prec\end{small}) $\rightarrow$ $g$ \\
\cmidrule(r){2-3}
& \multicolumn{2}{p{\defnwidth}}{
  Returns the cosine, sine, and tangent of $f$, respectively.
}
\end{tabular}
\newline\newline

\begin{tabular}{p{\methwidth} l r}
\toprule
\textbf{sec} & & $f$.sec(\begin{small}rnd\_mode = GMP\_RNDN, res\_prec=$f$.prec\end{small}) $\rightarrow$ $g$ \\
\textbf{csc} & & $f$.csc(\begin{small}rnd\_mode = GMP\_RNDN, res\_prec=$f$.prec\end{small}) $\rightarrow$ $g$ \\
\textbf{cot} & & $f$.cot(\begin{small}rnd\_mode = GMP\_RNDN, res\_prec=$f$.prec\end{small}) $\rightarrow$ $g$ \\
\cmidrule(r){2-3}
& \multicolumn{2}{p{\defnwidth}}{
  Returns the secant, cosecant, and cotangent of $f$, respectively.
}
\end{tabular}
\newline\newline

\begin{tabular}{p{\methwidth} l r}
\toprule
\textbf{acos} & & $f$.acos(\begin{small}rnd\_mode = GMP\_RNDN, res\_prec=$f$.prec\end{small}) $\rightarrow$ $g$ \\
\textbf{asin} & & $f$.asin(\begin{small}rnd\_mode = GMP\_RNDN, res\_prec=$f$.prec\end{small}) $\rightarrow$ $g$ \\
\textbf{atan} & & $f$.atan(\begin{small}rnd\_mode = GMP\_RNDN, res\_prec=$f$.prec\end{small}) $\rightarrow$ $g$ \\
\cmidrule(r){2-3}
& \multicolumn{2}{p{\defnwidth}}{
  Returns the arc-cosine, arc-sine, and arc-tangent of $f$, respectively.
}
\end{tabular}
\newline\newline

\begin{tabular}{p{\methwidth} l r}
\toprule
\textbf{cosh} & & $f$.cosh(\begin{small}rnd\_mode = GMP\_RNDN, res\_prec=$f$.prec\end{small}) $\rightarrow$ $g$ \\
\textbf{sinh} & & $f$.sinh(\begin{small}rnd\_mode = GMP\_RNDN, res\_prec=$f$.prec\end{small}) $\rightarrow$ $g$ \\
\textbf{tanh} & & $f$.tanh(\begin{small}rnd\_mode = GMP\_RNDN, res\_prec=$f$.prec\end{small}) $\rightarrow$ $g$ \\
\cmidrule(r){2-3}
& \multicolumn{2}{p{\defnwidth}}{
  Returns the hyperbolic cosine, sine, and tangent of $f$, respectively.
}
\end{tabular}
\newline\newline

\begin{tabular}{p{\methwidth} l r}
\toprule
\textbf{sech} & & $f$.sech(\begin{small}rnd\_mode = GMP\_RNDN, res\_prec=$f$.prec\end{small}) $\rightarrow$ $g$ \\
\textbf{csch} & & $f$.csch(\begin{small}rnd\_mode = GMP\_RNDN, res\_prec=$f$.prec\end{small}) $\rightarrow$ $g$ \\
\textbf{coth} & & $f$.coth(\begin{small}rnd\_mode = GMP\_RNDN, res\_prec=$f$.prec\end{small}) $\rightarrow$ $g$ \\
\cmidrule(r){2-3}
& \multicolumn{2}{p{\defnwidth}}{
  Returns the hyperbolic secant, cosecant, and cotangent of $f$, respectively.
}
\end{tabular}
\newline\newline

\begin{tabular}{p{\methwidth} l r}
\toprule
\textbf{acosh} & & $f$.acosh(\begin{small}rnd\_mode = GMP\_RNDN, res\_prec=$f$.prec\end{small}) $\rightarrow$ $g$ \\
\textbf{asinh} & & $f$.asinh(\begin{small}rnd\_mode = GMP\_RNDN, res\_prec=$f$.prec\end{small}) $\rightarrow$ $g$ \\
\textbf{atanh} & & $f$.atanh(\begin{small}rnd\_mode = GMP\_RNDN, res\_prec=$f$.prec\end{small}) $\rightarrow$ $g$ \\
\cmidrule(r){2-3}
& \multicolumn{2}{p{\defnwidth}}{
  Returns the hyperbolic arc-cosine, arc-sine, and arc-tangent of $f$, respectively.
}
\end{tabular}
\newline\newline

\begin{tabular}{p{\methwidth} l r}
\toprule
\textbf{log1p} & & $f$.log1p(\begin{small}rnd\_mode = GMP\_RNDN, res\_prec=$f$.prec\end{small}) $\rightarrow$ $g$ \\
\cmidrule(r){2-3}
& \multicolumn{2}{p{\defnwidth}}{
  Returns the logarithm of 1 plus $f$.
}
\end{tabular}
\newline\newline

\begin{tabular}{p{\methwidth} l r}
\toprule
\textbf{expm1} & & $f$.expm1(\begin{small}rnd\_mode = GMP\_RNDN, res\_prec=$f$.prec\end{small}) $\rightarrow$ $g$ \\
\cmidrule(r){2-3}
& \multicolumn{2}{p{\defnwidth}}{
  Returns the exponential of $f$ minus 1.
}
\end{tabular}
\newline\newline

\begin{tabular}{p{\methwidth} l r}
\toprule
\textbf{eint} & & $f$.eint(\begin{small}rnd\_mode = GMP\_RNDN, res\_prec=$f$.prec\end{small}) $\rightarrow$ $g$ \\
\cmidrule(r){2-3}
& \multicolumn{2}{p{\defnwidth}}{
  Returns the exponential integral of $f$. For positive $f$, the exponential integral is
  the sum of Euler's constant, of the logarithm of $f$, and of the sum for $k$ from $1$
  to infinity of $f$ to the power $k$, divided by $k$ and factorial($k$). For negative
  $f$, this method returns NaN.
}
\end{tabular}
\newline\newline

\begin{tabular}{p{\methwidth} l r}
\toprule
\textbf{li2} & & $f$.li2(\begin{small}rnd\_mode = GMP\_RNDN, res\_prec=$f$.prec\end{small}) $\rightarrow$ $g$ \\
\cmidrule(r){2-3}
& \multicolumn{2}{p{\defnwidth}}{
  Returns the real part of the dilogarithm of $f$. MPFR defines the dilogarithm as the
  integral of $-\log(1-t)/t$ from 0 to $f$.
}
\end{tabular}
\newline\newline

\begin{tabular}{p{\methwidth} l r}
\toprule
\textbf{gamma} & & $f$.gamma(\begin{small}rnd\_mode = GMP\_RNDN, res\_prec=$f$.prec\end{small}) $\rightarrow$ $g$ \\
\cmidrule(r){2-3}
& \multicolumn{2}{p{\defnwidth}}{
  Returns the value of the Gamma function on $f$. When $f$ is a negative integer, this
  method returns NaN.
}
\end{tabular}
\newline\newline

\begin{tabular}{p{\methwidth} l r}
\toprule
\textbf{lngamma} & & $f$.lngamma(\begin{small}rnd\_mode = GMP\_RNDN, res\_prec=$f$.prec\end{small}) $\rightarrow$ $g$ \\
\cmidrule(r){2-3}
& \multicolumn{2}{p{\defnwidth}}{
  Returns the value of the logarithm of the Gamma function on $f$. When
  $-2k-1 \leq f \leq -2k$, $k$ being a non-negative integer, this method returns NaN.
}
\end{tabular}
\newline\newline

\begin{tabular}{p{\methwidth} l r}
\toprule
\textbf{digamma} & & $f$.digamma(\begin{small}rnd\_mode = GMP\_RNDN, res\_prec=$f$.prec\end{small}) $\rightarrow$ $g$ \\
\cmidrule(r){2-3}
& \multicolumn{2}{p{\defnwidth}}{
  Returns the value of the Digamma (sometimes called Psi) function on $f$. When $f$ is
  negative, this method returns NaN.\newline
  \newline
  Only available in MPFR version 3.0.0 or later.
}
\end{tabular}
\newline\newline

\begin{tabular}{p{\methwidth} l r}
\toprule
\textbf{zeta} & & $f$.zeta(\begin{small}rnd\_mode = GMP\_RNDN, res\_prec=$f$.prec\end{small}) $\rightarrow$ $g$ \\
\cmidrule(r){2-3}
& \multicolumn{2}{p{\defnwidth}}{
  Returns the value of the Riemann Zeta function on $f$.
}
\end{tabular}
\newline\newline

\begin{tabular}{p{\methwidth} l r}
\toprule
\textbf{erf}  & & $f$.erf(\begin{small}rnd\_mode = GMP\_RNDN, res\_prec=$f$.prec\end{small}) $\rightarrow$ $g$ \\
\textbf{erfc} & & $f$.erfc(\begin{small}rnd\_mode = GMP\_RNDN, res\_prec=$f$.prec\end{small}) $\rightarrow$ $g$ \\
\cmidrule(r){2-3}
& \multicolumn{2}{p{\defnwidth}}{
  Returns the value of the error function on $f$ (respectively the complementary error
  function on $f$).
}
\end{tabular}
\newline\newline

\begin{tabular}{p{\methwidth} l r}
\toprule
\textbf{j0} & & $f$.j0(\begin{small}rnd\_mode = GMP\_RNDN, res\_prec=$f$.prec\end{small}) $\rightarrow$ $g$ \\
\textbf{j1} & & $f$.j1(\begin{small}rnd\_mode = GMP\_RNDN, res\_prec=$f$.prec\end{small}) $\rightarrow$ $g$ \\
\textbf{jn} & & $f$.jn(\begin{small}rnd\_mode = GMP\_RNDN, res\_prec=$f$.prec\end{small}) $\rightarrow$ $g$ \\
\cmidrule(r){2-3}
& \multicolumn{2}{p{\defnwidth}}{
  Returns the value of the first kind Bessel function of order $0$ (respectively $1$ and
  $n$) on $f$. When $f$ is NaN, this method returns NaN. When $f$ is +Inf or -Inf, this
  method returns $+0$. When $f$ is zero, this method returns +Inf or -Inf, depending on
  the parity and sign of $n$, and the sign of $f$.
}
\end{tabular}
\newline\newline

\begin{tabular}{p{\methwidth} l r}
\toprule
\textbf{y0} & & $f$.y0(\begin{small}rnd\_mode = GMP\_RNDN, res\_prec=$f$.prec\end{small}) $\rightarrow$ $g$ \\
\textbf{y1} & & $f$.y1(\begin{small}rnd\_mode = GMP\_RNDN, res\_prec=$f$.prec\end{small}) $\rightarrow$ $g$ \\
\textbf{yn} & & $f$.yn(\begin{small}rnd\_mode = GMP\_RNDN, res\_prec=$f$.prec\end{small}) $\rightarrow$ $g$ \\
\cmidrule(r){2-3}
& \multicolumn{2}{p{\defnwidth}}{
  Returns the value of the second kind Bessel function of order $0$ (respectively $1$ and
  $n$) on $f$. When $f$ is NaN or negative, this method returns NaN. When $f$ is +Inf,
  this method returns $+0$. When $f$ is zero, this method returns +Inf or -Inf, depending
  on the parity and sign of $n$.
}
\end{tabular}


\newpage
\section{Random Number Functions}

\subsection{Random State Initialization}

\begin{tabular}{p{\methwidth} l r}
\toprule
\textbf{new} & & GMP::RandState.new $\rightarrow$ \textit{mersenne twister state} \\
& & GMP::RandState.new(:default) $\rightarrow$ \textit{mersenne twister state} \\
& & GMP::RandState(:mt) $\rightarrow$ \textit{mersenne twister random state} \\
& & GMP::RandState.new(:lc\_2exp, a, c, m2exp) $\rightarrow$ \textit{linear congruential state} \\
& & GMP::RandState.new(:lc\_2exp\_size, size) $\rightarrow$ \textit{linear congruential state} \\
\cmidrule(r){2-3}
& \multicolumn{2}{p{\defnwidth}}{
  This method creates a new \gmprandstate instance. The first argument defaults
  to \textit{:default} (also \textit{:mt}), which initializes the \gmprandstate for a Mersenne
  Twister algorithm. No other arguments should be given if \textit{:default} or \textit{:mt} is
  specified.
  \newline\newline
  If the first argument given is \textit{:lc\_2exp}, then the \gmprandstate is
  initialized for a linear congruential algorithm. \textit{:lc\_2exp} must be followed with $a$,
  $c$, and $m2exp$. The algorithm can then proceed as ($X = (a*X + c) \mod 2^{m2exp}$).
  \newline\newline
  \gmprandstate can also be initialized for a linear congruential algorithm with
  \textit{:lc\_2exp\_size}. This initializer instead takes just one argument, $size$. $a$, $c$,
  and $m2exp$ are then chosen from a table, with $m2exp/2 > size$. The maximum size
  currently supported is 128.\newline

  \texttt{GMP::RandState.new \newline
          GMP::RandState.new(:mt) \newline
          GMP::RandState.new(:lc\_2exp, 1103515245, 12345, 15) \quad \#=> Perl's old rand() \newline
          GMP::RandState.new(:lc\_2exp, 25\_214\_903\_917, 11, 48) \quad \#=> drand48}
}
\end{tabular}

\subsection{Random State Seeding}

\begin{tabular}{p{\methwidth} l r}
\toprule
\textbf{seed} & & $state$.seed($integer$) $\rightarrow$ $integer$ \\
\cmidrule(r){2-3}
& \multicolumn{2}{p{\defnwidth}}{
  Set an initial seed value into $state$. $integer$ can be an instance of \gmpz,
  $Fixnum$, or $Bignum$.
}
\end{tabular}

\subsection{Integer Random Numbers}

\begin{tabular}{p{\methwidth} l r}
\toprule
\textbf{urandomb} & & $state$.urandomb($n$) $\rightarrow$ $integer$ \\
\cmidrule(r){2-3}
& \multicolumn{2}{p{\defnwidth}}{
  Generates a uniformly distributed random integer in the range $0$ to $2^n -1$,
  inclusive.
}
\end{tabular}
\newline\newline

\begin{tabular}{p{\methwidth} l r}
\toprule
\textbf{urandomm} & & $state$.urandomm($n$) $\rightarrow$ $integer$ \\
\cmidrule(r){2-3}
& \multicolumn{2}{p{\defnwidth}}{
  Generates a uniformly distributed random integer in the range $0$ to $n -1$,
  inclusive. $n$ can be an instance of \gmpz, $Fixnum$, or $Bignum$.
}
\end{tabular}

\subsection{Floating-Point Random Numbers (MPFR only)}

\begin{tabular}{p{\methwidth} l r}
\toprule
\textbf{mpfr\_urandomb} & & $state$.mpfr\_urandomb() $\rightarrow$ $floating-point$ \\
& & $state$.mprf\_urandomb($prec$) $\rightarrow$ $floating-point$ \\
\cmidrule(r){2-3}
& \multicolumn{2}{p{\defnwidth}}{
  Generates a uniformly distributed random float in the between $0$ and $1$.
  More precisely, the number can be seen as a float with a random non-normalized
  significand and exponent 0, which is then normalized (thus if $e$ denotes the
  exponent after normalization, then the least $-e$ significant bits of the
  significand are always 0).

  Optionally pass $prec$, the precision of the resultant GMP::F number.
}
\end{tabular}
\newline\newline

\begin{tabular}{p{\methwidth} l r}
\toprule
\textbf{mpfr\_urandom} & & $state$.mpfr\_urandom() $\rightarrow$ $integer$ \\
& & $state$.mprf\_urandom($rnd\_mode$) $\rightarrow$ $floating-point$ \\
& & $state$.mprf\_urandom($rnd\_mode$, $prec$) $\rightarrow$ $floating-point$ \\
\cmidrule(r){2-3}
& \multicolumn{2}{p{\defnwidth}}{
  Generate a uniformly distributed random float. The floating-point number
  can be seen as if a random real number is generated according to the continuous
  uniform distribution on the interval [0, 1] and then rounded in the direction
  $rnd$.

  Optionally pass $rnd\_mode$, a rounding mode.

  Also optionally pass $prec$, the precision of the resultant GMP::F number.
}
\end{tabular}
\newline\newline

\subsection{Floating-point Miscellaneous Functions (MPFR only)}

\begin{tabular}{p{\methwidth} l r}
\toprule
\textbf{mpfr\_buildopt\_tls\_p} & & GMP::F.mpfr\_buildopt\_tls\_p() $\rightarrow$ $integer$ \\
\cmidrule(r){2-3}
& \multicolumn{2}{p{\defnwidth}}{
  Available only in MPFR 3.0.0 and greater.
  \newline\newline
  From the MPFR Manual: Return a non-zero value if MPFR was compiled as thread safe using compiler-level Thread Local Storage (that is, MPFR was built with the \texttt{--enable-thread-safe} configure option, see INSTALL file), return zero otherwise.
}
\end{tabular}
\newline\newline

\begin{tabular}{p{\methwidth} l r}
\toprule
\textbf{mpfr\_buildopt\_decimal\_p} & & GMP::F.mpfr\_buildopt\_decimal\_p() $\rightarrow$ $integer$ \\
\cmidrule(r){2-3}
& \multicolumn{2}{p{\defnwidth}}{
  Available only in MPFR 3.0.0 and greater.
  \newline\newline
  From the MPFR Manual: Return a non-zero value if MPFR was compiled with decimal float support (that is, MPFR was built with the \texttt{--enable-decimal-float} configure option), return zero otherwise.
}
\end{tabular}

\newpage
\section{Benchmarking}

Benchmark results can be found in performance.pdf.

\end{document}
