\documentclass[pdftex,10pt]{article}
\usepackage[pdftex]{graphicx}
\setlength{\parindent}{0pt}
\newcommand{\HRule}{\rule{\linewidth}{0.8mm}}
\newcommand{\sectionline}{\rule{1.0\linewidth}{2.0pt}}
\usepackage[top=1.2in, bottom=0.8in, left=0.8in, right=0.8in]{geometry}
\setcounter{tocdepth}{4}
\title{gmp - Ruby bindings to the GMP library}
\date{September 27, 2009}
\author{Sam Rawlins}
\begin{document}
\huge{gmp}\\
\HRule
\begin{flushright}
\large{Ruby bindings to the GMP library}\\
\large{Edition 0.1.5}\\
\large{27 September 2009}\\
\end{flushright}
\vfill
\large{written by Sam Rawlins}\\
\large{with excessive quoting from the GMP Manual}
\newpage

\vfill
This manual describes how to use the gmp Ruby gem, which provides bindings to
the GNU multiple precision arithmetic library, version 4.3.x.\\
\\
Copyright 2009 Sam Rawlins.\\
No license yet.
\newpage

\tableofcontents
\newpage

\section{Introduction to GNU MP}

This entire page is copied verbatim from the GMP Manual.\\
\\
GNU MP is a portable library written in C for arbitrary precision arithmetic on
integers, rational numbers, and floating-point numbers. It aims to provide the
fastest possible arithmetic for all applications that need higher precision
than is directly supported by the basic C types.\\
\\
Many applications use just a few hundred bits of precision; but some
applications may need thousands or even millions of bits. GMP is designed to
give good performance for both, by choosing algorithms based on the sizes of
the operands, and by carefully keeping the overhead at a minimum.\\
\\
The speed of GMP is achieved by using fullwords as the basic arithmetic type,
by using sophisticated algorithms, by including carefully optimized assembly
code for the most common inner loops for many different CPUs, and by a general
emphasis on speed (as opposed to simplicity or elegance).\\
\\
There is assembly code for these CPUs: ARM, DEC Alpha 21064, 21164, and 21264,
AMD 29000, AMD K6, K6-2, Athlon, and Athlon64, Hitachi SuperH and SH-2, HPPA
1.0, 1.1, and 2.0, Intel Pentium, Pentium Pro/II/III, Pentium 4, generic x86,
Intel IA-64, i960, Motorola MC68000, MC68020, MC88100, and MC88110,
Motorola/IBM PowerPC 32 and 64, National NS32000, IBM POWER, MIPS R3000, R4000,
SPARCv7, SuperSPARC, generic SPARCv8, UltraSPARC, DEC VAX, and Zilog Z8000.
Some optimizations also for Cray vector systems, Clipper, IBM ROMP (RT), and
Pyramid AP/XP.\\
\\
For up-to-date information on GMP, please see the GMP web pages at\\
\\
\hangindent=0.5cm \texttt{http://gmplib.org/}\\

The latest version of the library is available at\\
\\
\texttt{ftp://ftp.gnu.org/gnu/gmp/}\\
\\
Many sites around the world mirror '\texttt{ftp.gnu.org}', please use a mirror
near you, see \texttt{http://www.gnu.org/order/ftp.html} for a full list.\\
\\
There are three public mailing lists of interest. One for release
announcements, one for general questions and discussions about usage of the GMP
library, and one for bug reports. For more information, see\\
\\
\texttt{http://gmplib.org/mailman/listinfo/}.\\
\\
The proper place for bug reports is gmp-bugs@gmplib.org. See Chapter 4
[Reporting Bugs], page 28 for information about reporting bugs.

\section{Introduction to the gmp gem}

The gmp Ruby gem is a Ruby library that provides bindings to GMP. The gem is
incomplete, and will likely only include a subset of the GMP functions. It is
built as a C extension for ruby, interacting with gmp.h. The gmp gem is not
endorsed or supported by GNU or the GMP team. The gmp gem also does not ship
with GMP, so GMP must be compiled separately.

\section{Installing the gmp gem}

\subsection{Prerequisites}
OK. First, we've got a few requirements. To install the gmp gem, you need one
of the following versions of Ruby:
\begin{itemize}
  \item (MRI) Ruby 1.8.6
  \item (MRI) Ruby 1.8.7 - tested most often.
  \item (MRI) Ruby 1.9.1 - more of a "release candidate" state. Please report
    bugs.
\end{itemize}
As you can see only Matz's Ruby Interpreter (MRI) is supported. I haven't even
put a thought into trying other interpreters/VMs.\\

Next is the platform, the combination of the architecture (processor) and OS.
As far as I can tell, if you can compile GMP on a given platform, you can use
the gmp gem there too.\\

Lastly, GMP. GMP must be compiled and working. "and working" means you ran "make check" while installing GMP. The following versions of GMP have been tested:
\begin{itemize}
  \item GMP 4.3.1
\end{itemize}

That's all. I don't intend to test any older versions, maybe 4.3.0 for completeness.\\

Here is a table of the exact environments I have tested the gmp gem on:\\
\\
\begin{tabular}{|l|r|r|} \hline
  Platform & Ruby & GMP \\ \hline
  Cygwin on x86   & Ruby 1.8.7 & GMP 4.3.1 \\
  Mac OS X 10.5.7 & Ruby 1.8.6 & GMP 4.3.1 \\ \hline
\end{tabular}

\subsection{Installing}
You may clone the gmp gem's git repository with:\\
\\
\hangindent=0.5cm \texttt{git clone git://github.com/srawlins/gmp.git}\\

Or you may install the gem from github:\\\\

\texttt{gem install srawlins-gmp}\\\\

At this time, the gem does not self-compile. To compile the C extensions, do
the following:\\\\

\hangindent=0.5cm \texttt{cd <srawlins-gmp gem directory>/ext\\
  ruby extconf.rb\\
  make}\\

There shouldn't be any errors, or warnings.

\section{GMP and gmp gem basics}

\end{document}